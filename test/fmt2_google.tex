
\documentclass[UTF8]{article}
\usepackage{xeCJK}
\usepackage{amsmath,amssymb}
\begin{document}

因为当  $D=0$  时 DS 方程是代数的,我们可以分析地推导出这种渐近行为:我们代入  $G_{2 n}=(-1)^{n+1}(2 n-1) ! g_{2 n}$  ,将  $2 n$  th DS 方程乘以  $x^{2 n}$  ,从  $n=1$  到  $\infty$  求和,并定义生成函数  $u(x) \equiv x g_2+x^3 g_4+x^5 g_6+\cdots$  。  $u(x)$  满足的微分方程是非线性的:  $$u^{\prime \prime}(x)=3 u^{\prime}(x) u(x)-u^3(x)-x
$$  其中  $u(0)=0$  和  $u^{\prime}(0)=G_2$  。我们通过代入  $u(x)=-y^{\prime}(x) / y(x)$  线性化 (5) 并得到  $y^{\prime \prime \prime}(x)=x y(x)$  ,其中  $y(0)=1, y^{\prime}(0)=0, y^{\prime \prime}(0)=-G_2$  。满足这些初始条件的精确解是  $$y(x)=\frac{2 \sqrt{2}}{\Gamma(1 / 4)} \int_0^{\infty} d t \cos (x t) e^{-t^4 / 4} .
$$  如果  $y(x)=0$  ,生成函数  $u(x)$  变为无穷大,因此  $|x|$  的最小值  $y(x)=0$  是  $u(x)$  级数的收敛半径。一个简单的图显示  $y(x)$  在  $x_0= \pm 2.4419682 \ldots$  处消失。 [9].因此,  $r=1 / x_0=0.409506 \ldots$  证实了 (4)。

(4) 中的渐近行为表明  $G_{2 n}$  比  $\gamma_{2 n}$  增长得更快,因为  $n \rightarrow \infty$  :  $$
\gamma_{2 n}=\frac{\int_{-\infty}^{\infty} d x x^{2 n} e^{-x^4 / 4}}{\int_{-\infty}^{\infty} d x e^{-x^4 / 4}} \sim 2^n \frac{\Gamma(n / 2+1 / 4)}{\Gamma(1 / 4)} .
$$ 

\end{document}

