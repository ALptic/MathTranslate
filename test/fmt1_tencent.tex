
\documentclass[UTF8]{article}
\usepackage{xeCJK}
\usepackage{amsmath,amssymb}
\begin{document}




因为当\( D=0 \)时DS方程是代数的,所以我们可以解析地推导出这种渐近行为:我们用\( G_{2 n}=(-1)^{n+1}(2 n-1) ! g_{2 n} \)代替\( 2 n \),将\( 2 n \) DS方程乘以\( x^{2 n} \),从\( n=1 \)求和到\( \infty \),并定义生成函数\( u(x) \equiv x g_{2}+x^{3} g_{4}+x^{5} g_{6}+\cdots \)。由\( u(x) \)满足的微分方程是非线性的:




\[u^{\prime \prime}(x)=3 u^{\prime}(x) u(x)-u^{3}(x)-x
\]其中\( u(0)=0 \)和\( u^{\prime}(0)=G_{2} \)。我们通过替换\( u(x)=-y^{\prime}(x) / y(x) \)来线性化(5),得到\( y^{\prime \prime \prime}(x)=x y(x) \),其中\( y(0)=1, y^{\prime}(0)=0, y^{\prime \prime}(0)=-G_{2} \)。满足这些初始条件的精确解为




\[y(x)=\frac{2 \sqrt{2}}{\Gamma(1 / 4)} \int_{0}^{\infty} d t \cos (x t) e^{-t^{4} / 4} .
\]


如果是\( y(x)=0 \),则母函数\( u(x) \)为无穷大,因此是\( |x| \)的最小值,此时\( y(x)=0 \)是\( u(x) \)的级数的收敛半径。一个简单的曲线图表明,\( y(x) \)在\( x_{0}= \pm 2.4419682 \ldots \)[9]处消失。因此,\( r=1 / x_{0}=0.409506 \ldots \),证实了(4)。



(4)中的渐近行为表明,\( G_{2 n} \)比\( \gamma_{2 n} \)增长得更快,因为\( n \rightarrow \infty \):


\[
\gamma_{2 n}=\frac{\int_{-\infty}^{\infty} d x x^{2 n} e^{-x^{4} / 4}}{\int_{-\infty}^{\infty} d x e^{-x^{4} / 4}} \sim 2^{n} \frac{\Gamma(n / 2+1 / 4)}{\Gamma(1 / 4)} .
\]


\end{document}

