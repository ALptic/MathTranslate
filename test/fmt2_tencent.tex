
\documentclass[UTF8]{article}
\usepackage{xeCJK}
\usepackage{amsmath,amssymb}
\begin{document}


因为当$D=0$时DS方程是代数的,我们可以解析地得到这个渐近行为:我们用$G_{2 n}=(-1)^{n+1}(2 n-1) ! g_{2 n}$代替$2 n$,用$x^{2 n}$乘以$2 n$的DS方程,从$n=1$求和到$\infty$,并定义母函数$u(x) \equiv x g_2+x^3 g_4+x^5 g_6+\cdots$。$u(x)$所满足的微分方程是非线性的:$$u^{\prime \prime}(x)=3 u^{\prime}(x) u(x)-u^3(x)-x
$$,其中$u(0)=0$和$u^{\prime}(0)=G_2$。我们通过代换$u(x)=-y^{\prime}(x) / y(x)$来线性化(5),得到$y^{\prime \prime \prime}(x)=x y(x)$,其中$y(0)=1, y^{\prime}(0)=0, y^{\prime \prime}(0)=-G_2$。满足这些初始条件的精确解是$$y(x)=\frac{2 \sqrt{2}}{\Gamma(1 / 4)} \int_0^{\infty} d t \cos (x t) e^{-t^4 / 4} .
$$,如果$y(x)=0$,母函数$u(x)$变为无穷大,则$|x|$的最小值$y(x)=0$是$u(x)$的级数的收敛半径。一个简单的曲线图表明,$y(x)$在$x_0= \pm 2.4419682 \ldots$处消失。[9]。因此,$r=1 / x_0=0.409506 \ldots$,证实了(4)。


文(4)中的渐近行为表明,$G_{2 n}$比$\gamma_{2 n}$增长得快得多,因为$n \rightarrow \infty$:$$
\gamma_{2 n}=\frac{\int_{-\infty}^{\infty} d x x^{2 n} e^{-x^4 / 4}}{\int_{-\infty}^{\infty} d x e^{-x^4 / 4}} \sim 2^n \frac{\Gamma(n / 2+1 / 4)}{\Gamma(1 / 4)} .
$$

\end{document}

